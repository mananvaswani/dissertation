\documentclass[11pt]{article}

% Packages included
\usepackage[utf8]{inputenc}
\usepackage{amsmath, amssymb, amsthm}
\usepackage{tikz, pgf}
\usepackage{capt-of}
\usetikzlibrary{quotes,angles,arrows}
\usepackage{mathrsfs}
\usepackage{float}
\usepackage[nottoc]{tocbibind}
\usepackage{mathtools}
\usepackage{complexity}
\usepackage{algorithm}
%\usepackage{algorithmic}
\usepackage[noend]{algpseudocode}
\usepackage{listings}
\usepackage{minted}
\usepackage{graphicx}
\usepackage[export]{adjustbox}


% Theorems, Definitions, Corollaries etc,

\theoremstyle{theorem}
\newtheorem{theorem}{Theorem}[section]

%\theoremstyle{theorem}
%\newtheorem{lemma}[theorem]{Lemma}
 
\theoremstyle{remark}
\newtheorem*{remark}{Remark}

%\theoremstyle{note}
%\newtheorem*{note}{Note}

\theoremstyle{plain}
\newtheorem{definition}[theorem]{Definition}% reset theorem numbering for each chapter

\theoremstyle{definition}
\newtheorem{example}[theorem]{Example}

\linespread{1}

% Special operators
\DeclareMathOperator*{\Var}{\mathrm{Var}}
\DeclareMathOperator*{\Cov}{\mathrm{Cov}}
\DeclareMathOperator*{\Per}{\mathrm{Per}}

\usepackage[margin=3.5cm]{geometry}

\begin{document}
\begin{titlepage}
    \begin{center}
        \vspace*{\fill}
        
        \Huge
        \textbf{A multi-core CPU implementation of the classical Boson Sampling algorithm}
        
        \LARGE
        
        \vspace{2cm}
        \textbf{Manan Vaswani}
        
        \vfill
        
        Level M\\
        40cp project
        
        \vspace{0.8cm}
        
        
        \Large
        Supervisor: Dr. Rapha\"el Clifford\\
	Date: 6th May, 2019
        
    \end{center}
\end{titlepage}

\newpage
\tableofcontents

\section{Conclusion}
The final results produced by us showed a speed up of around 200 times with the gcc compiler, and 20 times for the Intel compiler, as compared to our initial measurments. We also ran a few tests for $n=35$, and it took roughly 320 seconds to run. While these numbers may not seem impressive compared to the benchmark on the Tianhe-2 supercomputer \cite{wu2018}, our implementation uses a maximum of 28 cores, whereas the benchmark for $n=50$ computed in 600 minutes was made using 312?,000 cores. We project that in order to break that benchmark, we would need access to a supercomputer with only $\approx 10,000$ cores.
\subsubsection{Further Works}
%Testing for correctness
%Running it on larger supercomputer
%Running it on multiple processors
%Other optimsations
%Running it on a GPU

\bibliographystyle{unsrt}
\bibliography{project_bib}

\end{document}